%%%%%%%%%%%%%%%%%%%%%%%%%%%%%%%%%%%%%%%%%%%%
% Author: Bijesh Mishra              %
% License: GNU General Public License v3.0 %
%%%%%%%%%%%%%%%%%%%%%%%%%%%%%%%%%%%%%%%%%%%%
\documentclass[11pt]{beamer}
%The full list of themes is:: AnnArbor, Antibes, Bergen, Berkeley, Berlin, CambridgeUS, Copenhagen, Darmstadt, Dresden, Frankfurt,  Goettingen, Hannover, Ilmenau, JuanLesPins,  Luebeck, Madrid, Malmoe, Marburg, Montpellier, PaloAlto, Pittsburgh, Rochester, Singapore, Szeged, Warsaw, boxes, default
\usetheme{Singapore} % Use this with different color fot this presentation.
\usefonttheme[]{serif}
%\usetheme{Dresden} % I like this. Similar to Ilmenau. Szeged
%\usetheme{Copenhagen}
%\usetheme{Warsaw} % Same as Luebeck, Malmoe
% The full list of color themes is: default, albatross, beaver, beetle, crane, dolphin, dove, fly, lily, orchid, rose, seagull, seahorse, whale, wolverine
%\usecolortheme{}
\usecolortheme{dolphin} % with  Dresden.
\usepackage{graphicx}
\usepackage{fancyhdr}
\usepackage{hyperref}
\usepackage{anyfontsize}
\usepackage{tcolorbox}
\usepackage[utf8]{inputenc}
\usepackage{xcolor}
\usepackage{colortbl}
\newcommand{\RowColor}{\rowcolor{gray!60}}
\usepackage{adjustbox}
\usepackage{tikzsymbols}
\usepackage{array}
\usepackage{amsmath}
\usepackage{listings}
\usepackage{fontawesome5}
\usepackage{comment}

\lstset
{
	language={[LaTeX]TeX},
	frame=single,
	framesep=\fboxsep,
	framerule=\fboxrule,
	rulecolor=\color{black},
	xleftmargin=\dimexpr\fboxsep+\fboxrule,
	xrightmargin=\dimexpr\fboxsep+\fboxrule,
	breaklines=true,
	basicstyle=\small\tt,
	keywordstyle=\color{blue}\sf,
	commentstyle=\color{purple!100},
	backgroundcolor=\color{gray!10},
	tabsize=2,
	columns=flexible,
}

\graphicspath{ {./Images/} }
\usenavigationsymbolstemplate{}
\title{\bf Avoiding Plagiarism in Research}
%\subtitle{Conference}
\author{\bf Bijesh Mishra, Ph.D.}
\institute{\color{blue}
{\color{blue}\href{https://bijeshmishra.wordpress.com/}{\faIcon{wordpress} https://bijeshmishra.wordpress.com}} \\
\smallskip
{\color{blue}\href{https://twitter.com/bijubjs}{\faIcon{twitter} @bijubjs}}
{\color{blue}\href{https://www.facebook.com/BMishraPhD}{\faIcon{facebook} Dr. Bijesh Mishra}} 
{\color{blue}\href{https://www.linkedin.com/in/bijubjs/}{\faIcon{linkedin} @bijubjs}} \\
\smallskip
\faIcon{envelope} bzm0094@auburn.edu; bjs.misra@gmail.com \\
\vspace{0.5 cm}
Auburn University \\
%\faIcon{map-marker-alt}
Auburn, AL\\
\vspace{0.25cm}} 
\date{\today}

%%%%%%%%%%%%%%%%%%%%%%%%%%%%%%%%%%%%%%%%%%%%%%%%%%%%%%%%%%%%%%%%%%%
%%%%%%%%%%%%%%%%%%%%%%%%%%%%%%%%%%%%%%%%%%%%%%%%%%%%%%%%%%%%%%%%%%%
\begin{document}
\begin{frame} %[plain]
	{\frame{\titlepage}} 
\end{frame}
%%%%%%%%%%%%%%%%%%%%%%%%%%%%%%%%%%%%%%%%%%%%%%%%%%%%%%%%%%%%%%%%%%%%%

%%%%%%%%%%%%%%%%%%%%%%%%%%%%%%%%%%%%%%%%%%%%%%%%%%%%%%%%%%%%%%%%%%%%%

%	\section[Table of Contents]{}

%%%%%%%%%%%%%%%%%%%%%%%%%%%%%%%%%%%%%%%%%%%%%%%%%%%%%%%%%%%%%%%%%%%%%%
\begin{frame}{Table of Contents}
	\tableofcontents
\end{frame}
%%%%%%%%%%%%%%%%%%%%%%%%%%%%%%%%%%%%%%%%%%%%%%%%%%%%%%%%%%%%%%%%%%%%%%%
	
\section{Plagiarism:}
	
%%%%%%%%%%%%%%%%%%%%%%%%%%%%%%%%%%%%%%%%%%%%%%%%%%%%%%%%%%%%%%%%%%%%%%
\subsection{What is Plagiarism?}
	\begin{frame}{What is Plagiarism?}
%		 \tableofcontents[currentsection]
	\begin{itemize}
		\item {\color{blue}\href{https://ossja.ucdavis.edu/avoiding-plagiarism-mastering-art-scholarship}{Office of Student Support and Judicial Affairs, UC DAVIS: }} \textquotedblleft Plagiarism is using another's \underline{\textbf{works}} without giving \underline{\textbf{credit}}.\textquotedblright \\
		\vspace{0.25 cm}
		\item {\color{blue}\href{https://sites.auburn.edu/admin/universitypolicies/Policies/AcademicHonestyCode.pdf}{Auburn University (2019) Student Academic Honesty Code: }} Plagiarism is \textquotedblleft using words or ideas of another as one's own (page 100).\textquotedblright \\ 
		\vspace{0.25 cm}
		\item The U.S. Federal Definition from {\color{blue}\href{https://ori.hhs.gov/content/chapter-2-research-misconduct-office-science-and-technology-policy}{US Department of Health \& Human Service (HHS), The Office of Science and Technology Policy (OSTP):}} \textquotedblleft Plagiarism is the appropriation of another person’s ideas, processes, results, or words without giving appropriate credit.\textquotedblright \\
		\end{itemize}
	\end{frame}
%%%%%%%%%%%%%%%%%%%%%%%%%%%%%%%%%%%%%%%%%%%%%%%%%%%%%%%%%%%%%%%

%%%%%%%%%%%%%%%%%%%%%%%%%%%%%%%%%%%%%%%%%%%%%%%%%%%%%%%%%%%%%%%%%%%%%%
\subsection{When Plagiarism Occurs?}
\begin{frame}{When Plagiarism Occurs?}
		Two Common ways:
		\begin{itemize}
		\item Improper use of someone else's \textbf{works} or  \textbf{ideas}:
			\begin{itemize}
				\item Misappropriating published or unpublished works fully or partially.
				\smallskip
				\item Misappropriating concepts, ideas, theories, texts, image, data, etc.
			\end{itemize}
		
		\vspace{0.10 cm}
		\item Improper use of someone else's words.
			\begin{itemize}
				\item Using source information too closely when \textbf{paraphrasing}.
				\smallskip
				\item Use original content by substituting/shuffling words and \textbf{{cite}}.
				\smallskip
				\item Failing to enclose \textquotedblleft borrowed verbatim in quotation mark\textquotedblright.
				\smallskip
				\item Hiring someone to write your assignment, paper, thesis, etc.
		\end{itemize}
		\end{itemize}
\end{frame}
%%%%%%%%%%%%%%%%%%%%%%%%%%%%%%%%%%%%%%%%%%%%%%%%%%%%%%%%%%%%%%%

%%%%%%%%%%%%%%%%%%%%%%%%%%%%%%%%%%%%%%%%%%%%%%%%%%%%%%%%%%%%%%%%%%%%%%
\subsection{Why Plagiarism Occurs?}
\begin{frame}{Why Plagiarism Occurs?}
	Two Potential Reasons:
	\begin{itemize}
		\item Conscious Avoidance:
		\begin{itemize}
			\item Creating false citation.
			\item Use original content as it was with citation.
			\item Paraphrased properly in own words without citation.
		\end{itemize}
		\smallskip
		\item Unconscious Negligence:
		\begin{itemize}
			\item I know so everyone knows.
			\item Forget to acknowledge or cite.
		\end{itemize}
	\end{itemize}
	\textit{\textbf{\color{blue} Notes:}} \\
	\begin{itemize}
	\item Self-plagarism and text recycling might be taken as plagiarism.
	\item Falsification and fabrications are also research misconducts.
	\end{itemize}
\end{frame}
%%%%%%%%%%%%%%%%%%%%%%%%%%%%%%%%%%%%%%%%%%%%%%%%%%%%%%%%%%%%%%%

\section{Avoiding Plagiarism:}
%%%%%%%%%%%%%%%%%%%%%%%%%%%%%%%%%%%%%%%%%%%%%%%%%%%%%%%%%%%%%%%
\subsection{How to Avoid Plagiarism?}
\begin{frame}{How to Avoid Plagiarism?}
	\begin{itemize}
	\item Summarize, paraphrase, and cite.
	\smallskip
	\item Contact original idea/work generator and seek approval if needed.
	\smallskip
	\item Clearly and unambiguously differentiate your ideas/works from others.
	\smallskip
	\item Use {\color{gray}\textbf {common knowledge (?)}}.
	\smallskip
	\item Plagiarism Dilemma: \textbf{revise, rewrite,} and \textbf{cite}.	
\end{itemize}
\end{frame}

\subsection{Common Knowledge:}
\begin{frame}{What are \textquotedblleft Common Knowledge\textquotedblright?}
	\begin{itemize}
		\item Depends upon writer, reader, knowledge level, situation, events, consequence, relative field of work, etc.
		\smallskip
		\item Universal truth such as \textquotedblleft Sun rises from East.\textquotedblright
		\smallskip
		\item \textquotedblleft Alfred Marshall's Principle of Economics developed a supply-and-demand curve\textquotedblright \textbf{MAY BE} a common knowledge for \textbf{economics student} but \textbf{NOT} for \textbf{Palentobiology student}.
		\smallskip
		\item Find frequently used but uncited information (common knowledge) in established literature in your field and context.
	\end{itemize}
\end{frame}
%%%%%%%%%%%%%%%%%%%%%%%%%%%%%%%%%%%%%%%%%%%%%%%%%%%%%%%%%%%%%%%

%%%%%%%%%%%%%%%%%%%%%%%%%%%%%%%%%%%%%%%%%%%%%%%%%%%%%%%%%%%%%%%%%%%%%%
\subsection{Actions to Reduce Plagiarism:}
\begin{frame}{Actions to Reduce Plagiarism:}
	\begin{itemize}				
		\item National Science Foundation (NSF) Office of the Inspector General (OIG) can subpoena if plagiarism is suspected {\color{blue}\href{https://pubmed.ncbi.nlm.nih.gov/31324124/}{(Kornfeld, 2019)}}.
		\smallskip
		\item NSF Grant recipient faculties (88\%) were twice more likely to be guilty of research misconduct to that from National Institute of Health (NIH) (42\%) {\color{blue}\href{https://pubmed.ncbi.nlm.nih.gov/31324124/}{(Kornfeld, 2019)}}.
		\smallskip		
		\item Five {\color{blue}\href{https://oig.nsf.gov/investigations/research-misconduct/by-the-numbers}{NSF}} grant recipients, out of 15 concluded investigations, were guilty of plagiarism from October 2021 to September 2022.
		\smallskip
		\item {\color{blue}\href{		https://www.govinfo.gov/content/pkg/FR-2022-03-17/pdf/2022-05659.pdf}{Research misconducts conducted in the United States of America (US) are publicly archived in US National Archive Federal Register Library}}.
	\end{itemize}
\end{frame}
%%%%%%%%%%%%%%%%%%%%%%%%%%%%%%%%%%%%%%%%%%%%%%%%%%%%%%%%%%%%%%%

\section{Available Resources:}
%%%%%%%%%%%%%%%%%%%%%%%%%%%%%%%%%%%%%%%%%%%%%%%%%%%%%%%%%%%%%%%%%%%%%%
%\begin{comment}
\subsection{Resources in Nepal:}
\begin{frame}{Resources in Nepal:}
	\begin{itemize}
		\item Very little, lack of clear guidelines, virtually no resources in Nepali language (Ironically, my presentation is in English!).
		\item {\color{blue}\href{https://lawcommission.gov.np/en/}{Nepal Law Commission}}.
		\smallskip
		\item {\color{blue}\href{https://lawcommission.gov.np/en/?cat=373}{Copyright Act 2059 Bs (2002 AD)}}.
		\smallskip 
		\item {\color{blue} \href{http://www.nepalcopyright.gov.np/}{Nepal Copyright Registrar's Office, Government of Nepal}}.
		\smallskip
		\item {\color{blue} \href{https://www.nepjol.info/index.php/NJN/article/view/20516} {Roka, 2017. The recent trend of plagiarism in Nepal, \textit{Nepal Journal of Neuroscience}, 14(3)}}.
		\item Use citation management tools such as {\color{blue}\href{https://www.mendeley.com/}{Mendeley}}, {\color{blue}\href{https://www.zotero.org/}{Zotero}}, and {\color{blue}\href{https://endnote.com/}{Endnotes}} for efficient writing. Mendeley and zotero are free to use and as good as any other paid software.
	\end{itemize}
\end{frame}
%\end{comment}
%%%%%%%%%%%%%%%%%%%%%%%%%%%%%%%%%%%%%%%%%%%%%%%%%%%%%%%%%%%%%%%%%%%%%%

%%%%%%%%%%%%%%%%%%%%%%%%%%%%%%%%%%%%%%%%%%%%%%%%%%%%%%%%%%%%%%%%%%%%%%
\subsection{Auburn University:}
\begin{frame}{Auburn University:}
	\begin{itemize}
	\item {\color{blue}\href{	https://sites.auburn.edu/admin/universitypolicies/Policies/AcademicHonestyCode.pdf}{Auburn Univeristy Title XII, Academic Honesty Code}}.
	\vspace{0.2 cm}
	\item {\color{blue}\href{https://cws.auburn.edu/OVPR/pm/compliance/home}{Office of Research Compliance}}.
	\vspace{0.2 cm}
	\item {\color{blue}\href{https://cws.auburn.edu/OVPR/pm/compliance/rcr/home}{AU Basic CITI Responsible Conduct of Research (RCR) Training}}.
	\vspace{0.2 cm}
	\item {\color{blue}\href{https://auburn.edu/academic/provost/university-writing/miller-writing-center/}{Miller Writing Center}}.
	\vspace{0.2 cm}
	\item {\color{blue}\href{https://libguides.auburn.edu/plagiarismseminar}{Plagiarism: some information you should know}}.
	\vspace{0.2 cm}
	\item Flow chart showing the types and severity of plagiarism violations: {\color{blue}\href{https://thevisualcommunicationguy.com/wp-content/uploads/2014/09/Infographic_Did-I-Plagiarize1.jpg}{Did I plagiarize?}}
	\end{itemize}
\end{frame}
%%%%%%%%%%%%%%%%%%%%%%%%%%%%%%%%%%%%%%%%%%%%%%%%%%%%%%%%%%%%%%%%%%%%%%

%%%%%%%%%%%%%%%%%%%%%%%%%%%%%%%%%%%%%%%%%%%%%%%%%%%%%%%%%%%%%%%%%%%%%%
\subsection{Additional Resources:}
\begin{frame}{Additional Resources:}
\begin{itemize}
	\item {\color{blue}\href{https://ori.hhs.gov/}{HHS, The Office of Research Integrity}}.
	\smallskip
	\item {\color{blue}\href{https://www.nsf.gov/od/ogc/regulation.jsp}{NSF Regulation of Reserach}}.
	\smallskip
	\item {\color{blue}\href{https://oir.nih.gov/sourcebook/ethical-conduct/responsible-conduct-research-training}{NIH Responsible Conduct of Research Training}}.
	\smallskip
	\item {\color{blue}\href{https://about.citiprogram.org/}{Collaborative Institutional Training Initiatives (CITI Program)}}.
	\smallskip
	\item Plagiarism checker such as {\color{blue}\href{https://www.turnitin.com/}{TurnItIn}}, {\color{blue}\href{https://app.grammarly.com/}{Grammerly}}, etc.
	\smallskip
	\item AI and plagiarism: {\color{blue}\href{https://doi.org/10.1007/s12195-022-00754-8}{King and ChatGPT (2023). A conversation on artificial intelligence, chatbots, and plagiarism in higher education.\textit{Cel. Mol. Bioeng.}}}
\end{itemize}
\end{frame}
%%%%%%%%%%%%%%%%%%%%%%%%%%%%%%%%%%%%%%%%%%%%%%%%%%%%%%%%%%%%%%%%%%%%%%

\section{Summary:}
%%%%%%%%%%%%%%%%%%%%%%%%%%%%%%%%%%%%%%%%%%%%%%%%%%%%%%%%%%%%%%%%%%%%%%
\subsection{Summary:}
\begin{frame}{Summary:}
	\begin{itemize}
		\item Be creative.
		\vspace{0.2 cm}
		\item Give credit.
		\vspace{0.2 cm}
		\item Use common knowledge.
		\vspace{0.2 cm}
		\item Summarize, paraphrase, and cite.
		\vspace{0.2 cm}
		\item Revise, rewrite, and cite.	
	\end{itemize}
\end{frame}
%%%%%%%%%%%%%%%%%%%%%%%%%%%%%%%%%%%%%%%%%%%%%%%%%%%%%%%%%%%%%%%
\end{document}